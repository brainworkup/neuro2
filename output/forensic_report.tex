% Options for packages loaded elsewhere
% Options for packages loaded elsewhere
\PassOptionsToPackage{unicode}{hyperref}
\PassOptionsToPackage{hyphens}{url}
\PassOptionsToPackage{dvipsnames,svgnames,x11names}{xcolor}
%
\documentclass[
  letterpaper,
  DIV=11,
  numbers=noendperiod]{scrartcl}
\usepackage{xcolor}
\usepackage{amsmath,amssymb}
\setcounter{secnumdepth}{-\maxdimen} % remove section numbering
\usepackage{iftex}
\ifPDFTeX
  \usepackage[T1]{fontenc}
  \usepackage[utf8]{inputenc}
  \usepackage{textcomp} % provide euro and other symbols
\else % if luatex or xetex
  \usepackage{unicode-math} % this also loads fontspec
  \defaultfontfeatures{Scale=MatchLowercase}
  \defaultfontfeatures[\rmfamily]{Ligatures=TeX,Scale=1}
\fi
\usepackage{lmodern}
\ifPDFTeX\else
  % xetex/luatex font selection
\fi
% Use upquote if available, for straight quotes in verbatim environments
\IfFileExists{upquote.sty}{\usepackage{upquote}}{}
\IfFileExists{microtype.sty}{% use microtype if available
  \usepackage[]{microtype}
  \UseMicrotypeSet[protrusion]{basicmath} % disable protrusion for tt fonts
}{}
\makeatletter
\@ifundefined{KOMAClassName}{% if non-KOMA class
  \IfFileExists{parskip.sty}{%
    \usepackage{parskip}
  }{% else
    \setlength{\parindent}{0pt}
    \setlength{\parskip}{6pt plus 2pt minus 1pt}}
}{% if KOMA class
  \KOMAoptions{parskip=half}}
\makeatother
% Make \paragraph and \subparagraph free-standing
\makeatletter
\ifx\paragraph\undefined\else
  \let\oldparagraph\paragraph
  \renewcommand{\paragraph}{
    \@ifstar
      \xxxParagraphStar
      \xxxParagraphNoStar
  }
  \newcommand{\xxxParagraphStar}[1]{\oldparagraph*{#1}\mbox{}}
  \newcommand{\xxxParagraphNoStar}[1]{\oldparagraph{#1}\mbox{}}
\fi
\ifx\subparagraph\undefined\else
  \let\oldsubparagraph\subparagraph
  \renewcommand{\subparagraph}{
    \@ifstar
      \xxxSubParagraphStar
      \xxxSubParagraphNoStar
  }
  \newcommand{\xxxSubParagraphStar}[1]{\oldsubparagraph*{#1}\mbox{}}
  \newcommand{\xxxSubParagraphNoStar}[1]{\oldsubparagraph{#1}\mbox{}}
\fi
\makeatother


\usepackage{longtable,booktabs,array}
\usepackage{calc} % for calculating minipage widths
% Correct order of tables after \paragraph or \subparagraph
\usepackage{etoolbox}
\makeatletter
\patchcmd\longtable{\par}{\if@noskipsec\mbox{}\fi\par}{}{}
\makeatother
% Allow footnotes in longtable head/foot
\IfFileExists{footnotehyper.sty}{\usepackage{footnotehyper}}{\usepackage{footnote}}
\makesavenoteenv{longtable}
\usepackage{graphicx}
\makeatletter
\newsavebox\pandoc@box
\newcommand*\pandocbounded[1]{% scales image to fit in text height/width
  \sbox\pandoc@box{#1}%
  \Gscale@div\@tempa{\textheight}{\dimexpr\ht\pandoc@box+\dp\pandoc@box\relax}%
  \Gscale@div\@tempb{\linewidth}{\wd\pandoc@box}%
  \ifdim\@tempb\p@<\@tempa\p@\let\@tempa\@tempb\fi% select the smaller of both
  \ifdim\@tempa\p@<\p@\scalebox{\@tempa}{\usebox\pandoc@box}%
  \else\usebox{\pandoc@box}%
  \fi%
}
% Set default figure placement to htbp
\def\fps@figure{htbp}
\makeatother





\setlength{\emergencystretch}{3em} % prevent overfull lines

\providecommand{\tightlist}{%
  \setlength{\itemsep}{0pt}\setlength{\parskip}{0pt}}



 


\usepackage{booktabs}
\usepackage{longtable}
\usepackage{array}
\usepackage{multirow}
\usepackage{wrapfig}
\usepackage{float}
\usepackage{colortbl}
\usepackage{pdflscape}
\usepackage{tabu}
\usepackage{threeparttable}
\usepackage{threeparttablex}
\usepackage[normalem]{ulem}
\usepackage{makecell}
\usepackage{xcolor}
\KOMAoption{captions}{tableheading}
\makeatletter
\@ifpackageloaded{caption}{}{\usepackage{caption}}
\AtBeginDocument{%
\ifdefined\contentsname
  \renewcommand*\contentsname{Table of contents}
\else
  \newcommand\contentsname{Table of contents}
\fi
\ifdefined\listfigurename
  \renewcommand*\listfigurename{List of Figures}
\else
  \newcommand\listfigurename{List of Figures}
\fi
\ifdefined\listtablename
  \renewcommand*\listtablename{List of Tables}
\else
  \newcommand\listtablename{List of Tables}
\fi
\ifdefined\figurename
  \renewcommand*\figurename{Figure}
\else
  \newcommand\figurename{Figure}
\fi
\ifdefined\tablename
  \renewcommand*\tablename{Table}
\else
  \newcommand\tablename{Table}
\fi
}
\@ifpackageloaded{float}{}{\usepackage{float}}
\floatstyle{ruled}
\@ifundefined{c@chapter}{\newfloat{codelisting}{h}{lop}}{\newfloat{codelisting}{h}{lop}[chapter]}
\floatname{codelisting}{Listing}
\newcommand*\listoflistings{\listof{codelisting}{List of Listings}}
\makeatother
\makeatletter
\makeatother
\makeatletter
\@ifpackageloaded{caption}{}{\usepackage{caption}}
\@ifpackageloaded{subcaption}{}{\usepackage{subcaption}}
\makeatother
\usepackage{bookmark}
\IfFileExists{xurl.sty}{\usepackage{xurl}}{} % add URL line breaks if available
\urlstyle{same}
\hypersetup{
  pdftitle={NEUROCOGNITIVE EXAMINATION},
  colorlinks=true,
  linkcolor={blue},
  filecolor={Maroon},
  citecolor={Blue},
  urlcolor={Blue},
  pdfcreator={LaTeX via pandoc}}


\title{NEUROCOGNITIVE EXAMINATION}
\author{}
\date{}
\begin{document}
\maketitle


\subsection{Tests Administered}\label{tests-administered}

This section lists all tests administered during the evaluation.

\section{NEUROBEHAVIORAL STATUS
EXAMINATION}\label{neurobehavioral-status-examination}

\subsection{IDENTIFYING INFORMATION}\label{identifying-information}

Biggie is a 44-year-old, right-handed, English-speaking male who was
referred for a comprehensive neuropsychological evaluation by
Dr.~Referral Source. The evaluation was conducted over three sessions on
{[}dates of evaluation{]}. The purpose of the assessment was to evaluate
his current neurocognitive functioning in the context of a forensic
evaluation.

\subsection{REASON FOR REFERRAL}\label{reason-for-referral}

Biggie was referred for a comprehensive neuropsychological assessment to
evaluate his current cognitive functioning, assess for the presence of
any neurocognitive deficits, and determine how these findings may impact
his forensic case. Specific referral questions include:

\begin{enumerate}
\def\labelenumi{\arabic{enumi}.}
\tightlist
\item
  What is Biggie's current level of neurocognitive functioning?
\item
  Are there any neurocognitive deficits present that may have impacted
  his behavior or decision-making capacity?
\item
  How might these findings inform treatment recommendations and legal
  proceedings?
\end{enumerate}

\subsection{BACKGROUND INFORMATION}\label{background-information}

The background information was obtained through clinical interview with
Biggie, review of available records, and collateral information provided
by {[}collateral sources{]}.

\subsubsection{Medical History}\label{medical-history}

Biggie reported {[}brief medical history including any neurological
conditions, head injuries, loss of consciousness, seizures, chronic
medical conditions, surgeries, current medications{]}.

\subsubsection{Psychiatric History}\label{psychiatric-history}

Biggie reported {[}brief psychiatric history including any diagnoses,
hospitalizations, outpatient treatment, history of suicidal ideation or
attempts, substance use history{]}.

\subsubsection{Educational and Occupational
History}\label{educational-and-occupational-history}

Biggie completed {[}educational history{]}. His occupational history
includes {[}occupational history{]}.

\subsubsection{Legal History}\label{legal-history}

Biggie reported {[}brief legal history relevant to the current case{]}.

\subsubsection{Current Symptoms}\label{current-symptoms}

Biggie reported {[}current cognitive, emotional, behavioral symptoms{]}.

\subsection{ASSESSMENT PROCEDURES}\label{assessment-procedures}

The following assessment procedures were administered as part of this
evaluation:

\subsubsection{Clinical Interview and Behavioral
Observations}\label{clinical-interview-and-behavioral-observations}

\begin{itemize}
\tightlist
\item
  Comprehensive clinical interview
\item
  Behavioral observations during testing
\end{itemize}

\subsubsection{Neurocognitive Measures}\label{neurocognitive-measures}

\begin{itemize}
\tightlist
\item
  Wechsler Adult Intelligence Scale-Fifth Edition (WAIS-5)
\item
  Test of Premorbid Functioning (TOPF)
\item
  Neuropsychological Assessment Battery (NAB)
\item
  California Verbal Learning Test-3 Brief (CVLT3-Brief)
\item
  Rey-Osterrieth Complex Figure Test (ROCFT)
\item
  Delis-Kaplan Executive Function System (D-KEFS)
\item
  EXAMINER Executive Function Battery
\item
  Wechsler Individual Achievement Test-Fourth Edition (WIAT-4)
\end{itemize}

\subsubsection{Self-Report and Observer-Rated
Measures}\label{self-report-and-observer-rated-measures}

\begin{itemize}
\tightlist
\item
  Personality Assessment Inventory (PAI)
\item
  Conners Adult ADHD Rating Scales-2 (CAARS-2) Self and Observer Forms
\item
  Comprehensive Executive Function Inventory (CEFI) Self and Observer
  Forms
\end{itemize}

\subsubsection{Validity Measures}\label{validity-measures}

\begin{itemize}
\tightlist
\item
  PAI Validity Scales
\item
  Embedded performance validity measures
\end{itemize}

\section{BEHAVIORAL OBSERVATIONS}\label{behavioral-observations}

Biggie arrived on time for all testing sessions. He was appropriately
dressed and well-groomed. He was alert and oriented to person, place,
time, and situation. His gait was steady, and he did not require any
assistance with mobility. He maintained good eye contact throughout the
evaluation and demonstrated appropriate social skills and rapport.

Biggie's speech was fluent, clear, and of normal rate, rhythm, and
volume. His thought processes were logical, coherent, and goal-directed.
There was no evidence of hallucinations, delusions, or other thought
disturbances during the evaluation. His affect was appropriate to the
content of discussion, and he displayed a full range of emotional
expression.

Regarding test-taking behavior, Biggie demonstrated good attention and
concentration throughout the evaluation. He was able to maintain focus
on lengthy and complex tasks with minimal redirection. His effort and
motivation appeared consistently good across all testing sessions. He
approached challenging tasks with persistence and did not give up
easily. He responded well to encouragement when tasks became difficult.

Biggie's comprehension of test instructions was good, and he rarely
required repetition or clarification. His working style was methodical
and careful, with appropriate attention to detail. He demonstrated
appropriate frustration tolerance when faced with challenging tasks.
There were no notable behavioral observations that would suggest
invalidation of the test results.

The current test results are considered to be a valid representation of
Biggie's current neurocognitive functioning. Multiple embedded and
standalone performance validity measures were administered throughout
the evaluation, all of which were within normal limits, suggesting
adequate effort and engagement throughout testing.

\subsection{General Cognitive Ability}\label{sec-iq}

Biggie's overall intellectual functioning was assessed using the
Wechsler Adult Intelligence Scale-Fifth Edition (WAIS-5). His Full Scale
IQ (FSIQ) was in the Average range (Standard Score = 105, 63rd
percentile), indicating average overall intellectual abilities compared
to same-age peers. This score represents an integration of his
performance across verbal, nonverbal, and working memory domains.

Analysis of Biggie's index scores revealed a General Ability Index (GAI)
in the Average range (Standard Score = 107, 68th percentile), which
provides an estimate of general intellectual ability that is less
influenced by working memory and processing speed. His Verbal
Comprehension Index (VCI) was in the Average range (Standard Score =
108, 70th percentile), indicating average verbal reasoning, verbal
concept formation, and acquired knowledge. His Perceptual Reasoning
Index (PRI) was also in the Average range (Standard Score = 104, 61st
percentile), reflecting average nonverbal reasoning, visual-spatial
processing, and fluid intelligence.

Biggie's Working Memory Index (WMI) was in the Average range (Standard
Score = 102, 55th percentile), indicating average abilities in
attention, concentration, and mental manipulation of information. His
Processing Speed Index (PSI) was in the Low Average range (Standard
Score = 92, 30th percentile), suggesting somewhat reduced efficiency in
processing simple visual information quickly and accurately.

The Test of Premorbid Functioning (TOPF) was administered to estimate
Biggie's premorbid intellectual functioning. His performance on this
measure yielded a standard score of 110 (75th percentile), which is in
the High Average range. Comparison of his current FSIQ (105) with his
estimated premorbid functioning (110) suggests a slight decline of 5
points, which is not statistically significant and falls within normal
variability.

The Neuropsychological Assessment Battery (NAB) was also administered to
provide a comprehensive assessment of neurocognitive functioning.
Biggie's NAB Total Index score was in the Average range (Standard Score
= 103, 58th percentile), consistent with his WAIS-5 FSIQ. Analysis of
NAB domain scores revealed Average functioning across Attention
(Standard Score = 100, 50th percentile), Language (Standard Score = 105,
63rd percentile), Memory (Standard Score = 102, 55th percentile),
Spatial (Standard Score = 104, 61st percentile), and Executive Functions
(Standard Score = 101, 53rd percentile).

Overall, Biggie demonstrates average intellectual and neurocognitive
abilities across most domains, with a relative weakness in processing
speed. His pattern of scores is generally consistent with his
educational and occupational history, and there is no evidence of
significant decline from estimated premorbid functioning.

\subsection{verbal}\label{verbal}

Placeholder content for \_02-03\_verbal.qmd

\subsection{spatial}\label{spatial}

Placeholder content for \_02-04\_spatial.qmd

\subsection{memory}\label{memory}

Placeholder content for \_02-05\_memory.qmd

\subsection{executive}\label{executive}

Placeholder content for \_02-06\_executive.qmd

\subsection{adhd\_adult}\label{adhd_adult}

Placeholder content for \_02-09\_adhd\_adult.qmd

\subsection{emotion\_adult}\label{emotion_adult}

Placeholder content for \_02-10\_emotion\_adult.qmd

\subsection{sirf}\label{sirf}

Placeholder content for \_03-00\_sirf.qmd

\section{SUMMARY, INTERPRETATION, AND
RECOMMENDATIONS}\label{summary-interpretation-and-recommendations}

\subsection{Summary and
Interpretation}\label{summary-and-interpretation}

Biggie is a 44-year-old, right-handed, English-speaking male who was
referred for a comprehensive neuropsychological evaluation in the
context of a forensic assessment. The evaluation consisted of a clinical
interview, review of available records, behavioral observations, and
administration of standardized tests designed to assess various domains
of neurocognitive functioning.

\subsubsection{Neurocognitive
Functioning}\label{neurocognitive-functioning}

Biggie demonstrated average overall intellectual abilities, with a Full
Scale IQ in the Average range (SS = 105, 63rd percentile). His verbal
and nonverbal reasoning abilities were both in the Average range,
indicating balanced intellectual development. His working memory was
also average, though his processing speed was somewhat lower (Low
Average range), suggesting he may require additional time to process
information efficiently.

Analysis of specific cognitive domains revealed the following:

\begin{enumerate}
\def\labelenumi{\arabic{enumi}.}
\item
  \textbf{Attention/Executive Functioning}: Biggie demonstrated average
  attentional capacity and executive functioning abilities. He performed
  adequately on tasks requiring sustained attention, working memory,
  cognitive flexibility, planning, and problem-solving. His performance
  on measures of inhibitory control and set-shifting was within normal
  limits. Observer ratings indicated some mild difficulties with
  organization and time management in everyday settings, though these
  were not at a clinically significant level.
\item
  \textbf{Learning and Memory}: Biggie's learning and memory abilities
  were generally in the Average range. He demonstrated adequate
  encoding, storage, and retrieval of both verbal and visual
  information. His immediate and delayed recall were consistent,
  suggesting intact memory consolidation processes. There was no
  significant discrepancy between recognition and free recall,
  indicating adequate retrieval processes.
\item
  \textbf{Language}: Biggie's language abilities were intact, with
  Average performance on measures of verbal comprehension, expressive
  vocabulary, and verbal reasoning. His verbal fluency and confrontation
  naming were also within normal limits.
\item
  \textbf{Visuospatial Functioning}: Biggie demonstrated Average
  visuospatial processing abilities. His performance on tasks requiring
  visual perception, spatial relations, and constructional abilities was
  within normal limits.
\item
  \textbf{Motor Functioning}: Biggie's fine motor speed and dexterity
  were within normal limits bilaterally, with no significant differences
  between dominant and non-dominant hands.
\item
  \textbf{Emotional/Behavioral Functioning}: Self-report and
  observer-report measures indicated mild elevations in symptoms of
  anxiety and stress. There were no significant elevations on measures
  of depression, mania, psychosis, or personality pathology. Ratings of
  ADHD symptoms were within normal limits.
\end{enumerate}

\subsubsection{Validity of Results}\label{validity-of-results}

Multiple embedded and standalone performance validity measures were
administered throughout the evaluation, all of which were within normal
limits, suggesting adequate effort and engagement throughout testing.
The consistency of Biggie's performance across similar tasks also
supports the validity of these results. Therefore, the current test
results are considered to be a valid representation of his current
neurocognitive functioning.

\subsubsection{Clinical Impressions and Diagnostic
Considerations}\label{clinical-impressions-and-diagnostic-considerations}

Based on the comprehensive neuropsychological evaluation, Biggie
presents with intact neurocognitive functioning across all major
domains, with scores generally falling in the Average range. There is no
evidence of a neurocognitive disorder or other neuropsychological
impairment that would significantly impact his daily functioning or
decision-making capacity. His profile is consistent with his educational
and occupational history, and there is no evidence of significant
decline from estimated premorbid functioning.

The mild elevations in symptoms of anxiety and stress are consistent
with situational stressors related to his current legal situation and do
not appear to represent a primary psychiatric disorder. These symptoms
do not appear to significantly impact his cognitive functioning or
decision-making capacity.

\subsection{Recommendations}\label{recommendations}

\begin{enumerate}
\def\labelenumi{\arabic{enumi}.}
\tightlist
\item
  \textbf{Cognitive Strategies}: Despite Biggie's overall average
  cognitive functioning, his relatively lower processing speed suggests
  he may benefit from strategies to enhance efficiency in processing
  complex information. These could include:

  \begin{itemize}
  \tightlist
  \item
    Breaking down complex information into smaller, more manageable
    parts
  \item
    Using visual aids and written notes to supplement verbal information
  \item
    Allowing additional time for processing important information before
    making decisions
  \end{itemize}
\item
  \textbf{Stress Management}: Given the mild elevations in anxiety and
  stress symptoms, Biggie may benefit from:

  \begin{itemize}
  \tightlist
  \item
    Learning and implementing stress management techniques such as deep
    breathing, progressive muscle relaxation, and mindfulness meditation
  \item
    Regular physical exercise, adequate sleep, and proper nutrition to
    support overall brain health and stress resilience
  \item
    Considering short-term supportive counseling to address situational
    stressors related to his legal situation
  \end{itemize}
\item
  \textbf{Forensic Implications}: Based on the results of this
  evaluation:

  \begin{itemize}
  \tightlist
  \item
    There is no evidence of a neurocognitive disorder or other
    neuropsychological impairment that would have significantly impacted
    Biggie's decision-making capacity or behavior relevant to the legal
    issues at hand
  \item
    His cognitive profile suggests he would have been capable of
    understanding the nature and consequences of his actions
  \item
    He demonstrates adequate capacity to participate in legal
    proceedings, understand the charges against him, and assist in his
    defense
  \end{itemize}
\item
  \textbf{Follow-up}: A follow-up neuropsychological evaluation is not
  indicated at this time unless there are significant changes in
  Biggie's cognitive or emotional functioning.
\end{enumerate}

This report is based on the information available at the time of the
evaluation. If additional information becomes available that could
impact these findings and recommendations, a re-evaluation may be
warranted.

\section{RECOMMENDATIONS}\label{recommendations-1}

Based on the results of this comprehensive neuropsychological
evaluation, the following recommendations are provided:

\begin{enumerate}
\def\labelenumi{\arabic{enumi}.}
\tightlist
\item
  \textbf{Cognitive Strategies}: Despite Biggie's overall average
  cognitive functioning, his relatively lower processing speed suggests
  he may benefit from strategies to enhance efficiency in processing
  complex information:

  \begin{itemize}
  \tightlist
  \item
    Breaking down complex information into smaller, more manageable
    parts
  \item
    Using visual aids and written notes to supplement verbal information
  \item
    Allowing additional time for processing important information before
    making decisions
  \end{itemize}
\item
  \textbf{Stress Management}: Given the mild elevations in anxiety and
  stress symptoms, Biggie may benefit from:

  \begin{itemize}
  \tightlist
  \item
    Learning and implementing stress management techniques such as deep
    breathing, progressive muscle relaxation, and mindfulness meditation
  \item
    Regular physical exercise, adequate sleep, and proper nutrition to
    support overall brain health and stress resilience
  \item
    Considering short-term supportive counseling to address situational
    stressors related to his legal situation
  \end{itemize}
\item
  \textbf{Legal Proceedings}: Regarding his participation in legal
  proceedings:

  \begin{itemize}
  \tightlist
  \item
    Biggie demonstrates adequate capacity to understand the nature of
    legal proceedings, the charges against him, and to assist in his
    defense
  \item
    When presenting complex legal information, it may be helpful to
    provide written materials and allow additional time for processing
  \item
    Regular breaks during lengthy legal proceedings may help maintain
    optimal cognitive functioning
  \end{itemize}
\item
  \textbf{Follow-up Services}:

  \begin{itemize}
  \tightlist
  \item
    A follow-up neuropsychological evaluation is not indicated at this
    time unless there are significant changes in Biggie's cognitive or
    emotional functioning
  \item
    If Biggie experiences increased stress or anxiety symptoms, a
    referral for brief supportive counseling may be beneficial
  \item
    Educational materials about stress management and cognitive
    efficiency strategies should be provided
  \end{itemize}
\end{enumerate}

This report is based on the information available at the time of the
evaluation. If additional information becomes available that could
impact these findings and recommendations, a re-evaluation may be
warranted.

\subsection{signature}\label{signature}

Placeholder content for \_03-02\_signature.qmd

\subsection{appendix}\label{appendix}

Placeholder content for \_03-03\_appendix.qmd

\section{NEUROCOGNITIVE FINDINGS}\label{neurocognitive-findings}




\end{document}
